\chapter{Preface}
\section{Leserschaft}

Dieses Dokument richtet sich an die Kunden des Team Red, Kurs Software
Engeneering and Design Fr\"{u}hlingssemester 2018. Es dient als
Grundbeschreibung der Applikation welche durch das Team entwickelt wird.

\section{Version}

\begin{table}[h]
 \caption{Version}
 \begin{tabularx}{\textwidth}{|l|l|X|}
     \hline
     \textbf{Versionsnummer}  & \textbf{Autor}  & \textbf{Beschreibung} \\
     \hline
     1.0                      & Roccaro         & Grundfassung, Bassierend auf Design Thinking \\
     \hline
 \end{tabularx}
 \label{table: Version}
\end{table}

\section{Versionsbegr\"{u}ndungen}

\begin{table}[h]
 \caption{Versionsbegr\"{u}ndungen}
 \begin{tabularx}{\linewidth}{|l|l|X|}
     \hline
     \textbf{Versionsnummer}  & \textbf{Autor}  & \textbf{Beschreibung} \\
     \hline
     1.0                      & Roccaro         & Grundfassung des
     Spezifikationsdokuments, Fassung zur Ansicht der Spezifikation durch Kunden. \\
     \hline
 \end{tabularx}
 \label{table: Versionsbegr\"{u}ndungen}
\end{table}

\section{Versions\"{a}nderungen}

\begin{table}[h]
 \caption{Versions\"{a}nderungen}
 \begin{tabularx}{\textwidth}{|l|l|X|}
     \hline
     \textbf{Versionsnummer}  & \textbf{Autor}  & \textbf{Beschreibung} \\
     \hline
     1.0                      & Roccaro         & Grundfassung des
     Spezifikationsdokuments, Fassung zur Ansicht der Spezifikation durch Kunden. \\
     \hline
 \end{tabularx}
 \label{table: Versions\"{a}nderungen}
\end{table}
