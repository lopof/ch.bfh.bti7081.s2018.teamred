\chapter{System architecture}
Bei der Applikation handelt es sich um eine Webapplikation, sie soll also \gls{Cross Platform} funktionieren. Einzige Systemanforderung an den Client ist ein moderner Browser. Durch den steigenden Trend an mobilen Ger\"{a}ten soll die Applikation sowohl auf Smartphones wie auch Tablets funktionieren, dies geschieht durch ein \gls{responsive Design} und einer \gls{mobile first} Entwicklung. Die Business Logik der Applikation soll in Java geschrieben werden, da sich das Team bereits ein breites Wissen in dieser Technologie erarbeitet hat. Genauere Spezifikationen an das Backend und die Datenbank werden im Kapitel Appendices spezifiziert.

\section{\"{U}bersicht der System Architektur}

Die Webapplikation wird mit \gls{Vaadin} umgesetzt. Dabei wird ein hoher Wert auf Simplizit\"{a}t seitens Client gesetzt. S\"{a}mtliche Daten werden auf der Datenbank des Backends abgelegt. Auf der Clientseite werden keine Installationen durchgef\"{u}hrt. \\ \\
Alle Daten werden ausschliesslich von der Applikation verwendet und nicht Drittanbietern zur Verf\"{u}gung gestellt. Die Applikation geht mit den Personendaten voll \gls{GDPR compliant} um. \\ \\
Der Server selbst wird in einer Cloud positioniert, die Backups sowie fundamentale Sicherheitsmassnahmen werden durch den Hoster des Servers durchgef\"{u}hrt.
