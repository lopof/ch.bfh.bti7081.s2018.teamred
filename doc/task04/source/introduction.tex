\chapter{Introduction}

Das System wird ben\"{o}tigt im Umgang mit Angstst\"{o}rungen bzw. Sozialen Phobien. Aktuell existieren keine L\"{o}sungen f\"{u}r Patienten in ambulanter Behandlung die an Angstst\"{o}rungen leiden. Unsere Applikation wird den Patienten bei der Genesung unterst\"{u}tzen und im mehr Lebensqualit\"{a}t geben. Auch soll es dem Patienten erm\"{o}glichen durch weniger Angst wieder ein aktives Mitglied der Gesellschaft zu werden.

\section{Kurzbeschreibung der Funktionen des Systems}

Unsere Applikation bietet \"{u}ber \gls{Gameification} einen Anreiz sich jeden Tag schrittweise an ein Ziel anzun\"{a}hern. Die Applikation bietet die M\"{o}glichkeit, dass der Patient zusammen mit dem Therapeuten \gls{Challenges} erstellt. Diese \gls{Challenges} sind Aufgaben welche der Patient anschliessend gestellt bekommt. Nach der Ausf\"{u}hrung einer Challenge wird der Patient nach einer Beschreibung des Erlebten gefragt. Diese Bechreibungen sind f\"{u}r den Therapeuten zug\"{a}nglich, sp\"{a}ter w\"{a}hrend den Therapiesitzungen k\"{o}nnen sie besprochen werden.

\section{Wie wird es mit anderen Systemen zusammen arbeiten}

Die Applikation wird selbst laufen bzw. keinen Zugang zu anderen Systemen ben\"{o}tigen. Es sind keine Schnittstellen zu anderen Systemen f\"{u}r den Datenabgleich vorgesehen. \\ \\
Die Applikation kommuniziert nicht mit anderen Systemen. Die Verbindungen welche projektiert werden, sind Emails oder Push Nachrichten an den Therapeuten, welche von der Applikation ausgel\"{o}st werden.
